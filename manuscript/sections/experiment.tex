\section{Experiment}
\label{sec:experiment}
In this section we will introduce the experimental settings and results. Our GPU device is Tesla P100 with 16GB memory. We use PyTorch to implement all the codes.

\subsection{Dataset}
\label{sec:dataset}
The dataset of our experiment is  PHEME \cite{DBLP:conf/coling/KochkinaLZ18} and RumorEval\cite{DBLP:conf/semeval/EnayetE17}. The details of PHEME are shown in Table. \ref{tab:pheme} and the details of RumorEval are shown in Table. \ref{tab:RumorEval}

\begin{table}[tbp]
	\caption{PHEME}
	\centering
	\label{tab:pheme}
	\resizebox{1\linewidth}{!}{
		\begin{tabular}{|c|c|c|c|c|}
			\hline
			\textbf{Events} & \textbf{Definition} & \textbf{Threads} & \textbf{Branch} & \textbf{Tweets}\\
			\hline
			Charlie Hebdo & the set of tweets &wer&1&\\
			\hline
			Sydney siege & a tweet &wer&1&\\
			\hline
			Ferguson & the set of rumor events&wer&1&\\
			\hline
			Ottawa Shooting & a rumor event&wer&1&\\
			\hline		
			Germanwings-crash & a rumor event&wer&1&\\
			\hline			
		\end{tabular}
	}	
\end{table}

\begin{table}[tbp]
	\caption{RumorEval}
	\centering
	\label{tab:RumorEval}
	\resizebox{1\linewidth}{!}{
		\begin{tabular}{|c|c|c|c|c|}
			\hline
			\textbf{Events} & \textbf{Definition} & \textbf{Threads} & \textbf{Branch} & \textbf{Tweets}\\
			\hline
			Charlie Hebdo & the set of tweets &wer&1&\\
			\hline
			Sydney siege & a tweet &wer&1&\\
			\hline
			Ferguson & the set of rumor events&wer&1&\\
			\hline
			Ottawa Shooting & a rumor event&wer&1&\\
			\hline		
			Germanwings-crash & a rumor event&wer&1&\\
			\hline			
		\end{tabular}
	}	
\end{table}
