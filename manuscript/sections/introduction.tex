\section{Introduction}
\label{sec:introduction}
The high-speed development of social networks brings explosive scale of information. The convenience of the social networks accelerates the diffusion of information. People create and publish messages without any limitations, which leads to the widespread dissemination of rumors in social networks \cite{DBLP:journals/corr/KurkaGZ15, DBLP:journals/csur/ZubiagaABLP18, DBLP:conf/sirocco/KostkaOW08, vosoughi2018spread}. Rumors are usually with unjudged veracity and may include substantive misinformation, which causes huge damages to the society. For instance, a large number of rumors arise during the CONVID-19 epidemic, such as "More than ten thousand people die in Wuhan." and "Liquor can kill the virus.", causing great public panic. Consequently, defeating rumors is an urgent task for social networks. Considering the large volumes of data on social networks, manually defeating rumors is impossible. 

To automatically defeat rumors, a lot of studies have been carried out. Generally speaking, the contributions are divided into two categories:  diffusion-based rumor source identification and content-based rumor detection. Diffusion-based rumor source identification aims to locate the source of the rumor, by analyzing the infection status of the network with known topology \cite{DBLP:conf/sigmetrics/ShahZ10, DBLP:journals/tit/ShahZ11, DBLP:conf/kdd/LappasTGM10}. With rumor sources located, we can effectively minimize the negative influence at the early stage of rumor diffusion. Content-based rumor detection aims to judge the veracity of a series of microblogs that include the source microblog and its comments. As shown in Fig. this task is a pipeline that can be divided into 4 sub-tasks: rumor detection, rumor tracking, stance classification, and veracity \cite{DBLP:journals/csur/ZubiagaABLP18, DBLP:conf/coling/KochkinaLZ18}. Rumor detection is the most popular task and it attracts lots of attention, which aims to classifier whether a microblogs series is a rumor. There are many studies for it \cite{DBLP:conf/socinfo/ZubiagaLP17, DBLP:conf/www/Ma0W19,DBLP:conf/naacl/NguyenDCD19, DBLP:journals/corr/abs-1906-05659}. Stance classification is to judge the emotion of a piece of microblog text \cite{DBLP:conf/semeval/EnayetE17, DBLP:conf/semeval/X17a, DBLP:conf/coling/ZubiagaKLPL16}. The possible emotions include: support, deny, query, and comments. The goal of the rumor veracity task is to judge if a piece of microblog tell the truth, which is a binary classification problem \cite{DBLP:conf/coling/KochkinaLZ18, DBLP:conf/acl/LiZS19, DBLP:conf/acl/KumarC19}. All these three sub-tasks above are hot topics with many well-designed models for them. However, for rumor tracking task, the significant work is less.

Rumor tracking task is to collect the related microblogs of a given rumor event. Usually, it is transformed into a binary classification problem that to classify the posts to related or unrelated. Although there are several models for this task, some of the models are proposed a long time ago \cite{DBLP:conf/emnlp/QazvinianRRM11} and the others are not specifically for rumor tracking task \cite{DBLP:conf/www/ChengNB20}.
We believe there is still room for improvement if we design a specific model for rumor tracking. 

The main contributions of this work are summarized a follows:
\begin{itemize}
	\item We proposed an end-to-end deep learning model named xxxx to solve the rumor tracking task specifically.
	\item We find several vital features for rumor tracking tasks, such as xxxx. By utilizing these features, xxxx improves accuracy.
	\item We conduct experiments on benchmark datasets, and the experimental results show the superior of xxxx.
\end{itemize}

The rest of this article is structured as follows. In Section \ref{sec:related}, we introduce several important work related to xxxx. In Section \ref{sec:perliminary}, we introduce some notations and background knowledge of this work. Our proposed model xxxx will be introduce in Section \ref{sec:model}. Then, section \ref{sec:experiment} shows the experiments. Finally, we will give the conclusion and future work in Section \ref{sec:conclusion}.